%-----------------------------------------------------------------------------
% Exercicio 4
\section{Quarto Exercício: Polos e Zeros}

%-----------------------------------------------------------------------------
% 4.a
\subsection{}

Podemos escrever a função de transferência $H(z)$ associada ao filtro $y(n)$ como quociente de produtos que envolvem os valores dos zeros e dos pólos do filtro:

\[\displaystyle
\begin{array}{l c l}
  H(z) & = & a_0 \frac{\prod\limits_{i=1}^M(1-Z_i z^{-1})}{\prod\limits_{i=1}^N (1-P_i z^{-1})} \quad = \quad \frac{a_0 (1 - Z_1 z^{-1})(1 - Z_2 z^{-1} )}{(1 - P_1 z^{-1})(1 - P_2 z^{-1})} \quad = \\
       & = & \frac{a_0 (1+\sqrt{A}z^{-1})(1-\sqrt{A}z^{-1})}{(1-Ae^{i\theta}z^{-1})(1-Ae^{-i\theta}z^{-1})} \quad = \quad \frac{a_0 (1-Az^{-2})}{1-Ae^{i\theta}z^{-1} - Ae^{-i\theta}z^{-1} + A^2 e^{i\theta-i\theta}z^{-2}} \quad = \\
       & = & \frac{a_0 (1-Az^{-2})}{1-A(e^{i\theta}+e^{-i\theta})z^{-1}+A^2 z^{-2}} \quad = \frac{a_0 (1-Az^{-2})}{1-2Acos(\theta)z^{-1}+A^2 z^{-2}} \quad = \\
       & = & \frac{ \sum\limits_{k=0}^M a_k z^{-k}}{ 1 + \sum\limits_{k=1}^N b_k z^{-k} } \quad = \quad \frac{N(z)}{D(z)}
\end{array}
\]

Na relação acima, os coeficientes de $N(z)$ e $D(z)$ são, respectivamente, os coeficientes $a_i$ e $b_i$ do filtro $y(n)$. Se supusermos $a_0=1$, por conveniência nos cálculos, teremos:

\[\displaystyle
\begin{array}{l c l c l}
  N(z) & = & 1-Az^{-2} & \Rightarrow & a_1 = 0, \quad a_2 = -A \\
  D(z) & = & 1-2Acos(\theta)z^{-1}+A^2 z^{-2} & \Rightarrow & b_1 = -2Acos(\theta), \quad b2 = A^2
\end{array}
\]

E portanto, a equação do filtro fica:

\begin{equation}\displaystyle
  y(n) = x(n) - Ax(n-2) -2Acos(\theta)y(n-1) + A^2y(n-2)
\end{equation}


%-----------------------------------------------------------------------------
% 4.b
\subsection{}

Para encontrar o ganho de um filtro em uma certa frequência $\omega$, basta
calcular a magnitude da equação de transferência do filtro nesta frequência, ou
seja, medir $H(e^{i\omega})$. Ao substituir $z$ na equação por $e^{i\omega}$,
teremos termos da forma $e^{-i\omega k}$, onde $\omega$ representará uma
frequência em radianos/amostra.

Dada uma frequência $f$ em Hertz (ciclos/segundo) e um valor $R$ da taxa de
amostragem da entrada em amostras/segundo, podemos representá-la como $2 \pi f$
radianos/segundo, ou $\frac{2 \pi f}{R}$ radianos/amostra. Dessa forma, uma
frequência $f$ dentro da taxa de Nyquist será mapeada em uma frequência $\omega
= \frac{2 \pi f}{R}$ entre $-\pi$ e $\pi$. Então, supondo que o sinal de
entrada tenha passado por um filtro passa-baixo e que portanto não possui
frequências maiores que a taxa de Nyquist, definimos
$G(f):[\frac{-R}{2},\frac{R}{2}] \rightarrow \mathbb{R}^{\ge 0}$ com $G(f) =
|H(e^{\frac{i 2 \pi f}{R}})|$. Daí, temos:

\[\displaystyle
\begin{array}{l c l}
  G(f) & = & |H(e^{i\omega})| \quad = \quad \sqrt{H(e^{i\omega})H(e^{-i\omega})} \quad = \\
       & = & \sqrt{ \frac{1-Ae^{-i2\omega}}{1-2Acos(\theta)e^{-i\omega}+A^2 e^{-i2\omega}} \frac{1-Ae^{i2\omega}}{1-2Acos(\theta)e^{i\omega}+A^2 e^{i\omega}}} \quad = \\
       & = & \sqrt{\frac{1-A(e^{-2i\omega}+e^{i2\omega})+A^2e^{-i2\omega+i2\omega}}{}} \quad = \sqrt{\frac{1-2Acos(2\omega)+A^2}{}}\\
\end{array}
\]



%-----------------------------------------------------------------------------
% 4.c
\subsection{}


%-----------------------------------------------------------------------------
% 4.d
\subsection{}
