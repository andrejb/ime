%-----------------------------------------------------------------------------
% Exercicio 4
\section{Quarto Exercício: Polos e Zeros}

%-----------------------------------------------------------------------------
% 4.a
\subsection{}

Podemos escrever a função de transferência $H(z)$ associada ao filtro $y(n)$ como quociente de produtos que envolvem os valores dos zeros e dos pólos do filtro:

\[\displaystyle
\begin{array}{l c l}
  H(z) & = & a_0 \frac{\prod\limits_{i=1}^M(1-Z_i z^{-1})}{\prod\limits_{i=1}^N (1-P_i z^{-1})} \quad = \quad \frac{a_0 (1 - Z_1 z^{-1})(1 - Z_2 z^{-1} )}{(1 - P_1 z^{-1})(1 - P_2 z^{-1})} \quad = \\
       & = & \frac{a_0 (1+\sqrt{A}z^{-1})(1-\sqrt{A}z^{-1})}{(1-Ae^{i\theta}z^{-1})(1-Ae^{-i\theta}z^{-1})} \quad = \quad \frac{a_0 (1-Az^{-2})}{1-Ae^{i\theta}z^{-1} - Ae^{-i\theta}z^{-1} + A^2 e^{i\theta-i\theta}z^{-2}} \quad = \\
       & = & \frac{a_0 (1-Az^{-2})}{1-A(e^{i\theta}+e^{-i\theta})z^{-1}+A^2 z^{-2}} \quad = \frac{a_0 (1-Az^{-2})}{1-2Acos(\theta)z^{-1}+A^2 z^{-2}} \quad = \\
       & = & \frac{ \sum\limits_{k=0}^M a_k z^{-k}}{ 1 + \sum\limits_{k=1}^N b_k z^{-k} } \quad = \quad \frac{N(z)}{D(z)}
\end{array}
\]

Na relação acima, os coeficientes de $N(z)$ e $D(z)$ são, respectivamente, os coeficientes $a_i$ e $b_i$ do filtro $y(n)$. Se supusermos $a_0=1$, por conveniência nos cálculos, teremos:

\[\displaystyle
\begin{array}{l c l c l}
  N(z) & = & 1-Az^{-2} & \Rightarrow & a_1 = 0, \quad a_2 = -A \\
  D(z) & = & 1-2Acos(\theta)z^{-1}+A^2 z^{-2} & \Rightarrow & b_1 = -2Acos(\theta), \quad b2 = A^2
\end{array}
\]

E portanto, a equação do filtro fica:

\begin{equation}\displaystyle
  y(n) = x(n) - Ax(n-2) -2Acos(\theta)y(n-1) + A^2y(n-2)
\end{equation}


%-----------------------------------------------------------------------------
% 4.b
\subsection{}

Para encontrar o ganho de um filtro em uma certa frequência $\omega$, basta
calcular a magnitude da equação de transferência do filtro nesta frequência, ou
seja, medir $H(e^{i\omega})$. Ao substituir $z$ na equação por $e^{i\omega}$,
teremos termos da forma $e^{-i\omega k}$, onde $\omega$ representará uma
frequência em radianos/amostra.

Dada uma frequência $f$ em Hertz (ciclos/segundo) e um valor $R$ da taxa de
amostragem da entrada em amostras/segundo, podemos representá-la como $2 \pi f$
radianos/segundo, ou $\frac{2 \pi f}{R}$ radianos/amostra. Dessa forma, uma
frequência $f$ dentro da taxa de Nyquist será mapeada em uma frequência $\omega
= \frac{2 \pi f}{R}$ entre $-\pi$ e $\pi$. Então, supondo que o sinal de
entrada tenha passado por um filtro passa-baixo e que portanto não possui
frequências maiores que a taxa de Nyquist, definimos
$G(f):[\frac{-R}{2},\frac{R}{2}] \rightarrow \mathbb{R}^{\ge 0}$ com $G(f) =
|H(e^{\frac{i 2 \pi f}{R}})|$. Daí, temos:

\[\displaystyle
\begin{array}{l c l}
  G(f) & = &\displaystyle |H(e^{i\omega})| \quad = \quad \left| \frac{(1+\sqrt{A}e^{-i\omega})(1-\sqrt{A}e^{-i\omega})}{(1-Ae^{i\theta}e^{-i\omega})(1-Ae^{-i\theta}e^{-i\omega})} \right| \quad = \\
       & = &\displaystyle \quad \frac{\left|(1+\sqrt{A}e^{-i\omega})\right|\left|(1-\sqrt{A}e^{-i\omega})\right|}{\displaystyle\left|(1-Ae^{i\theta}e^{-i\omega})\right|\left|(1-Ae^{-i\theta}e^{-i\omega})\right|} \quad = \quad \frac{\left| H_1(e^{i\omega})\right| \left| H_2(e^{i\omega})\right|}{\displaystyle\left| H_3(e^{i\omega})\right|\left| H_4(e^{i\omega})\right|}
\end{array}
\]

Podemos calcular cada termo separadamente:

\[
\begin{array}{l c l}
  |H_1(e^{i\omega})| & = &\displaystyle \left|(1+\sqrt{A}cos(\omega) -i\sqrt{A}sen(\omega))\right| \quad = \\
                     & = & \sqrt{(1+\sqrt{A}cos(\omega))^2+(-\sqrt{A}sen(\omega))^2} \quad = \\
                     & = &\displaystyle \sqrt{1+2\sqrt{A}cos(\omega) + A} \\
  |H_2(e^{i\omega})| & = & \displaystyle \left|(1-\sqrt{A}cos(\omega) +i\sqrt{A}sen(\omega))\right| \quad = \\
                     & = & \sqrt{(1-\sqrt{A}cos(\omega))^2+(\sqrt{A}sen(\omega))^2} \quad = \\
                     & = &\displaystyle \sqrt{1-2\sqrt{A}cos(\omega) + A} \\
  |H_3(e^{i\omega})| & = &\displaystyle \left|(1-Acos(\theta-\omega)+iAsen(\theta-\omega))\right| \quad = \\
                     & = &\displaystyle \sqrt{(1-Acos(\theta-\omega))^2+(Asen(\omega-\theta))^2} \\
                     & = &\displaystyle \sqrt{1-2Acos(\theta-\omega)+A^2} \\
  |H_4(e^{i\omega})| & = &\displaystyle \left|(1-Acos(\theta+\omega)+iAsen(\theta+\omega))\right| \quad = \\
                     & = &\displaystyle \sqrt{(1-Acos(\theta+\omega))^2+(Asen(\omega+\theta))^2} \\
                     & = &\displaystyle \sqrt{1-2Acos(\theta+\omega)+A^2}
\end{array}
\]

E finalmente substituir $\omega = \frac{2 \pi f}{R}$, para obter a expressão final:

\begin{equation}
\begin{array}{l c l c l}
  G(f) = \left| H(e^{i\frac{2 \pi f}{R}}) \right| = \frac{\sqrt{1+2\sqrt{A}cos(\frac{2 \pi f}{R}) + A} \sqrt{1-2\sqrt{A}cos(\frac{2 \pi f}{R}) + A}}{\sqrt{1-2Acos(\theta-\frac{2 \pi f}{R})+A^2}\sqrt{1-2Acos(\theta+\frac{2 \pi f}{R})+A^2}}
\end{array}
\end{equation}



%-----------------------------------------------------------------------------
% 4.c
\subsection{}

\[
\begin{array}{l c l c l}
  G(\frac{R\theta}{2\pi}) & = & \left| H(e^{i\theta}) \right| \quad = \quad \frac{\sqrt{1+2\sqrt{A}cos(\theta) + A} \sqrt{1-2\sqrt{A}cos(\theta) + A}}{\sqrt{1-2Acos(\theta-\theta)+A^2}\sqrt{1-2Acos(\theta+\theta)+A^2}} \quad & \\
  & = & \frac{\sqrt{1+2\sqrt{A}cos(\theta)+A-2\sqrt{A}cos(\theta)-4Acos^2(\theta)-2\sqrt{A}Acos(\theta)+A+2A\sqrt{A}cos(\theta)+A^2}}{\sqrt{1-2A+A^2}\sqrt{1-2A(cos^2(\theta)-sen^2(\theta)+A^2)}} \quad = \\
  & = & \frac{\sqrt{1-4Acos^2(\theta)+2A+A^2}}{\sqrt{(1-A)^2} \sqrt{1-4Acos^2(\theta)+2A+A^2)}} = \frac{1}{1-A}
\end{array}
\]

\[
\begin{array}{l c l c l}
  G(0) & = & \left| H(e^{i0}) \right| \quad = \quad \frac{\sqrt{1+2\sqrt{A}cos(0) + A} \sqrt{1-2\sqrt{A}cos(0) + A}}{\sqrt{1-2Acos(\theta-0)+A^2}\sqrt{1-2Acos(\theta+0)+A^2}} \quad & \\
       & = & \frac{\sqrt{1+2\sqrt{A} + A} \sqrt{1-2\sqrt{A} + A}}{\sqrt{1-2Acos(\theta)+A^2}\sqrt{1-2Acos(\theta)+A^2}} \quad & \\
       & = & \frac{\sqrt{1-2A+A^2}}{1-2Acos(\theta)+A^2} \quad = \quad \frac{(1-A)}{1-2Acos(\theta)+A^2} \\
\end{array}
\]

\[
\begin{array}{l c l c l}
  G(\frac{R}{2}) & = & \left| H(e^{i\pi}) \right| \quad = \quad \frac{\sqrt{1+2\sqrt{A}cos(\pi) + A} \sqrt{1-2\sqrt{A}cos(\pi) + A}}{\sqrt{1-2Acos(\theta-\pi)+A^2}\sqrt{1-2Acos(\theta+\pi)+A^2}} \quad & \\
         & = & \frac{\sqrt{1-2\sqrt{A}+A} \sqrt{1+2\sqrt{A}+A}}{\sqrt{1-2Acos(\theta-\pi)+A^2}\sqrt{1-2Acos(\theta+\pi)+A^2}} \quad = \\
         & = & \frac{1-A}{1-2Acos(\theta+\pi)+A^2} \\
\end{array}
\]

Se $\theta = \frac{\pi}{2}$:

\[
\begin{array}{l c l c l}
  G(0) & = & \frac{(1-A)}{1-2Acos(\frac{\pi}{2})+A^2} \quad = \quad \frac{1-A}{1+A^2} \\
\end{array}
\]

\[
\begin{array}{l c l c l}
  G(\frac{R}{2}) & = & \frac{1-A}{1-2Acos(\frac{3\pi}{2})+A^2} \quad  = \quad \frac{1-A}{1+A^2} \\
\end{array}
\]

O gráfico de $G(f)$ com $\theta = \frac{\pi}{2}$ pode ser visto na figura \ref{fig:4c}. Podemos notar a influência simétrica em $f = \frac{44100}{2} = 22050$Hz.

\begin{figure}
  \fbox{\includegraphics[width=\textwidth]{grafico-4d-0.eps}}
  \caption{\small{$G(f)$ com $\theta = \frac{\pi}{2}$, $A = 0,8$ e $R = 44100$Hz.}}
  \label{fig:4c}
\end{figure}

%-----------------------------------------------------------------------------
% 4.d
\subsection{}

Os gráficos de $G(f)$ no intervalo $[\frac{-R}{2},\frac{R}{2}]$ com $\theta = \frac{\pi}{4}$ e $\theta = \frac{3\pi}{4}$podem ser vistos nas figuras \ref{fig:4d-1} e \ref{fig:4d-2}, respectivamente. Podemos notar em cada figura como a proximidade dos pólos dos pontos do círculo unitário com ângulo $\frac{\pm\pi}{4}$ e $\frac{\pm3\pi}{4}$ influenciam as frequências próximas de $\frac{R}{8} = 5012,5$ e $\frac{3R}{8} = 1537,5$. 

\begin{figure}
  \fbox{\includegraphics[width=\textwidth]{grafico-4d-1.eps}}
  \caption{\small{$G(f)$ com $\theta = \frac{\pi}{4}$, $A = 0,8$ e $R = 44100$Hz.}}
  \label{fig:4d-1}
\end{figure}

\begin{figure}
  \fbox{\includegraphics[width=\textwidth]{grafico-4d-2.eps}}
  \caption{\small{$G(f)$ com $\theta = \frac{3\pi}{4}$, $A = 0,8$ e $R = 44100$Hz.}}
  \label{fig:4d-2}
\end{figure}

Finalmente, se normalizamos a função de transferência por um fator de $(1-A)$, teremos:

\[\displaystyle
\begin{array}{l c l c l}
  H(z) & = & \frac{(1-A) (1-Az^{-2})}{1-2Acos(\theta)z^{-1}+A^2 z^{-2}}  & = & \frac{1 - Az^{-2}-A+A^2z^{-2}}{1-2Acos(\theta)z^{-1}+A^2 z^{-2}} \quad = \\
       & = & \frac{(1-A) + (A^2 - A)z^{-2}}{1-2Acos(\theta)z^{-1}+A^2 z^{-2}}  &\Rightarrow & a_0 = (1-A), \quad a_2 = (A^2 - A) \\
       & & & & b_1 = -2Acos(\theta), \quad b_2 = A^2
\end{array}
\]

E o filtro normalizado, será dado por:

\begin{equation}\displaystyle
  y(n) = (1-A)x(n) - (A^2 - A)x(n-2) -2Acos(\theta)y(n-1) + A^2y(n-2)
\end{equation}

